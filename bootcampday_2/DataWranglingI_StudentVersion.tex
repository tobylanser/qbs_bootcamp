% Options for packages loaded elsewhere
\PassOptionsToPackage{unicode}{hyperref}
\PassOptionsToPackage{hyphens}{url}
\documentclass[
]{article}
\usepackage{xcolor}
\usepackage[margin=1in]{geometry}
\usepackage{amsmath,amssymb}
\setcounter{secnumdepth}{-\maxdimen} % remove section numbering
\usepackage{iftex}
\ifPDFTeX
  \usepackage[T1]{fontenc}
  \usepackage[utf8]{inputenc}
  \usepackage{textcomp} % provide euro and other symbols
\else % if luatex or xetex
  \usepackage{unicode-math} % this also loads fontspec
  \defaultfontfeatures{Scale=MatchLowercase}
  \defaultfontfeatures[\rmfamily]{Ligatures=TeX,Scale=1}
\fi
\usepackage{lmodern}
\ifPDFTeX\else
  % xetex/luatex font selection
\fi
% Use upquote if available, for straight quotes in verbatim environments
\IfFileExists{upquote.sty}{\usepackage{upquote}}{}
\IfFileExists{microtype.sty}{% use microtype if available
  \usepackage[]{microtype}
  \UseMicrotypeSet[protrusion]{basicmath} % disable protrusion for tt fonts
}{}
\makeatletter
\@ifundefined{KOMAClassName}{% if non-KOMA class
  \IfFileExists{parskip.sty}{%
    \usepackage{parskip}
  }{% else
    \setlength{\parindent}{0pt}
    \setlength{\parskip}{6pt plus 2pt minus 1pt}}
}{% if KOMA class
  \KOMAoptions{parskip=half}}
\makeatother
\usepackage{color}
\usepackage{fancyvrb}
\newcommand{\VerbBar}{|}
\newcommand{\VERB}{\Verb[commandchars=\\\{\}]}
\DefineVerbatimEnvironment{Highlighting}{Verbatim}{commandchars=\\\{\}}
% Add ',fontsize=\small' for more characters per line
\usepackage{framed}
\definecolor{shadecolor}{RGB}{248,248,248}
\newenvironment{Shaded}{\begin{snugshade}}{\end{snugshade}}
\newcommand{\AlertTok}[1]{\textcolor[rgb]{0.94,0.16,0.16}{#1}}
\newcommand{\AnnotationTok}[1]{\textcolor[rgb]{0.56,0.35,0.01}{\textbf{\textit{#1}}}}
\newcommand{\AttributeTok}[1]{\textcolor[rgb]{0.13,0.29,0.53}{#1}}
\newcommand{\BaseNTok}[1]{\textcolor[rgb]{0.00,0.00,0.81}{#1}}
\newcommand{\BuiltInTok}[1]{#1}
\newcommand{\CharTok}[1]{\textcolor[rgb]{0.31,0.60,0.02}{#1}}
\newcommand{\CommentTok}[1]{\textcolor[rgb]{0.56,0.35,0.01}{\textit{#1}}}
\newcommand{\CommentVarTok}[1]{\textcolor[rgb]{0.56,0.35,0.01}{\textbf{\textit{#1}}}}
\newcommand{\ConstantTok}[1]{\textcolor[rgb]{0.56,0.35,0.01}{#1}}
\newcommand{\ControlFlowTok}[1]{\textcolor[rgb]{0.13,0.29,0.53}{\textbf{#1}}}
\newcommand{\DataTypeTok}[1]{\textcolor[rgb]{0.13,0.29,0.53}{#1}}
\newcommand{\DecValTok}[1]{\textcolor[rgb]{0.00,0.00,0.81}{#1}}
\newcommand{\DocumentationTok}[1]{\textcolor[rgb]{0.56,0.35,0.01}{\textbf{\textit{#1}}}}
\newcommand{\ErrorTok}[1]{\textcolor[rgb]{0.64,0.00,0.00}{\textbf{#1}}}
\newcommand{\ExtensionTok}[1]{#1}
\newcommand{\FloatTok}[1]{\textcolor[rgb]{0.00,0.00,0.81}{#1}}
\newcommand{\FunctionTok}[1]{\textcolor[rgb]{0.13,0.29,0.53}{\textbf{#1}}}
\newcommand{\ImportTok}[1]{#1}
\newcommand{\InformationTok}[1]{\textcolor[rgb]{0.56,0.35,0.01}{\textbf{\textit{#1}}}}
\newcommand{\KeywordTok}[1]{\textcolor[rgb]{0.13,0.29,0.53}{\textbf{#1}}}
\newcommand{\NormalTok}[1]{#1}
\newcommand{\OperatorTok}[1]{\textcolor[rgb]{0.81,0.36,0.00}{\textbf{#1}}}
\newcommand{\OtherTok}[1]{\textcolor[rgb]{0.56,0.35,0.01}{#1}}
\newcommand{\PreprocessorTok}[1]{\textcolor[rgb]{0.56,0.35,0.01}{\textit{#1}}}
\newcommand{\RegionMarkerTok}[1]{#1}
\newcommand{\SpecialCharTok}[1]{\textcolor[rgb]{0.81,0.36,0.00}{\textbf{#1}}}
\newcommand{\SpecialStringTok}[1]{\textcolor[rgb]{0.31,0.60,0.02}{#1}}
\newcommand{\StringTok}[1]{\textcolor[rgb]{0.31,0.60,0.02}{#1}}
\newcommand{\VariableTok}[1]{\textcolor[rgb]{0.00,0.00,0.00}{#1}}
\newcommand{\VerbatimStringTok}[1]{\textcolor[rgb]{0.31,0.60,0.02}{#1}}
\newcommand{\WarningTok}[1]{\textcolor[rgb]{0.56,0.35,0.01}{\textbf{\textit{#1}}}}
\usepackage{graphicx}
\makeatletter
\newsavebox\pandoc@box
\newcommand*\pandocbounded[1]{% scales image to fit in text height/width
  \sbox\pandoc@box{#1}%
  \Gscale@div\@tempa{\textheight}{\dimexpr\ht\pandoc@box+\dp\pandoc@box\relax}%
  \Gscale@div\@tempb{\linewidth}{\wd\pandoc@box}%
  \ifdim\@tempb\p@<\@tempa\p@\let\@tempa\@tempb\fi% select the smaller of both
  \ifdim\@tempa\p@<\p@\scalebox{\@tempa}{\usebox\pandoc@box}%
  \else\usebox{\pandoc@box}%
  \fi%
}
% Set default figure placement to htbp
\def\fps@figure{htbp}
\makeatother
\setlength{\emergencystretch}{3em} % prevent overfull lines
\providecommand{\tightlist}{%
  \setlength{\itemsep}{0pt}\setlength{\parskip}{0pt}}
\usepackage{bookmark}
\IfFileExists{xurl.sty}{\usepackage{xurl}}{} % add URL line breaks if available
\urlstyle{same}
\hypersetup{
  pdftitle={Data Wrangling I},
  pdfauthor={QBS Bootcamp 2025},
  hidelinks,
  pdfcreator={LaTeX via pandoc}}

\title{Data Wrangling I}
\author{QBS Bootcamp 2025}
\date{}

\begin{document}
\maketitle

\subsubsection{Lesson Objectives}\label{lesson-objectives}

\subparagraph{At the end of this lecture you should be able
to:}\label{at-the-end-of-this-lecture-you-should-be-able-to}

\begin{enumerate}
\def\labelenumi{\arabic{enumi}.}
\tightlist
\item
  Generate random data
\item
  Derive new variables and subset data frames based on your existing
  data
\item
  Move between long and wide formatted data
\item
  Produce basic boxplots and scatter plots using ggpubr
\item
  Perform basic manipulation of character strings
\end{enumerate}

\subsubsection{Resources}\label{resources}

Overview of ggpubr: \url{https://rpkgs.datanovia.com/ggpubr/}

Examples of reshape2 package:
\url{https://seananderson.ca/2013/10/19/reshape/}

Wide vs long data frames:
\url{https://www.youtube.com/watch?v=pHPgMNXyzqc} \(~\)

\subsubsection{Generating Random Data}\label{generating-random-data}

For this lecture, we will start by generating a new data set. To
generate random variables, we can use the \emph{rnorm} and \emph{rbinom}
functions.

\begin{Shaded}
\begin{Highlighting}[]
\CommentTok{\# Define a variable of random systolic blood pressure}
\FunctionTok{rnorm}\NormalTok{(}\AttributeTok{n =} \DecValTok{10}\NormalTok{,}\AttributeTok{mean =} \FloatTok{128.4}\NormalTok{,}\AttributeTok{sd =} \FloatTok{19.6}\NormalTok{)}
\end{Highlighting}
\end{Shaded}

\begin{verbatim}
##  [1] 128.5551 126.4302 131.5658 131.0775 117.4854 146.0162 105.4042 117.4629
##  [9] 155.6464 171.2996
\end{verbatim}

Did you get the same values as me? Why or why not?

\(~\)

\begin{Shaded}
\begin{Highlighting}[]
\FunctionTok{set.seed}\NormalTok{(}\DecValTok{123}\NormalTok{)}
\FunctionTok{rnorm}\NormalTok{(}\AttributeTok{n =} \DecValTok{10}\NormalTok{,}\AttributeTok{mean =} \FloatTok{128.4}\NormalTok{,}\AttributeTok{sd =} \FloatTok{19.6}\NormalTok{)}
\end{Highlighting}
\end{Shaded}

\begin{verbatim}
##  [1] 117.4147 123.8885 158.9507 129.7820 130.9340 162.0153 137.4340 103.6048
##  [9] 114.9377 119.6650
\end{verbatim}

\begin{Shaded}
\begin{Highlighting}[]
\FunctionTok{rnorm}\NormalTok{(}\AttributeTok{n =} \DecValTok{10}\NormalTok{,}\AttributeTok{mean =} \FloatTok{128.4}\NormalTok{,}\AttributeTok{sd =} \FloatTok{19.6}\NormalTok{)}
\end{Highlighting}
\end{Shaded}

\begin{verbatim}
##  [1] 152.3920 135.4524 136.2551 130.5694 117.5055 163.4235 138.1579  89.8543
##  [9] 142.1466 119.1333
\end{verbatim}

\begin{Shaded}
\begin{Highlighting}[]
\FunctionTok{set.seed}\NormalTok{(}\DecValTok{123}\NormalTok{)}
\FunctionTok{rnorm}\NormalTok{(}\AttributeTok{n =} \DecValTok{10}\NormalTok{,}\AttributeTok{mean =} \FloatTok{128.4}\NormalTok{,}\AttributeTok{sd =} \FloatTok{19.6}\NormalTok{)}
\end{Highlighting}
\end{Shaded}

\begin{verbatim}
##  [1] 117.4147 123.8885 158.9507 129.7820 130.9340 162.0153 137.4340 103.6048
##  [9] 114.9377 119.6650
\end{verbatim}

What do you think the set seed function does?

\(~\)

Many functions in R incorporate some level of randomization such as
sampling, some clustering algorithms, and some plotting functions.
Always be aware of if you are using randomization, and always ensure
that your code and results are reproducible by using the
\emph{set.seed()} function.

\(~\)

Now let's generate a random binary variable

\begin{Shaded}
\begin{Highlighting}[]
\FunctionTok{set.seed}\NormalTok{(}\DecValTok{103}\NormalTok{)}
\FunctionTok{rbinom}\NormalTok{(}\AttributeTok{n =} \DecValTok{10}\NormalTok{,}\AttributeTok{size =} \DecValTok{1}\NormalTok{,}\AttributeTok{prob =} \FloatTok{0.5}\NormalTok{)}
\end{Highlighting}
\end{Shaded}

\begin{verbatim}
##  [1] 0 0 1 1 0 0 0 0 0 0
\end{verbatim}

\begin{Shaded}
\begin{Highlighting}[]
\FunctionTok{rbinom}\NormalTok{(}\AttributeTok{n =} \DecValTok{10}\NormalTok{,}\AttributeTok{size =} \DecValTok{10}\NormalTok{,}\AttributeTok{prob =} \FloatTok{0.5}\NormalTok{)}
\end{Highlighting}
\end{Shaded}

\begin{verbatim}
##  [1] 5 4 6 6 7 3 3 4 4 4
\end{verbatim}

What does the ``size'' parameter change? If we were creating a binary
variable like biological sex, what would you set it as?

\(~\)

\begin{Shaded}
\begin{Highlighting}[]
\FunctionTok{set.seed}\NormalTok{(}\DecValTok{103}\NormalTok{)}
\FunctionTok{mean}\NormalTok{(}\FunctionTok{rbinom}\NormalTok{(}\AttributeTok{n =} \DecValTok{100}\NormalTok{,}\AttributeTok{size =} \DecValTok{10}\NormalTok{,}\AttributeTok{prob =} \FloatTok{0.5}\NormalTok{))}
\end{Highlighting}
\end{Shaded}

\begin{verbatim}
## [1] 5.01
\end{verbatim}

\begin{Shaded}
\begin{Highlighting}[]
\FunctionTok{mean}\NormalTok{(}\FunctionTok{rbinom}\NormalTok{(}\AttributeTok{n =} \DecValTok{100}\NormalTok{,}\AttributeTok{size =} \DecValTok{10}\NormalTok{,}\AttributeTok{prob =} \FloatTok{0.7}\NormalTok{))}
\end{Highlighting}
\end{Shaded}

\begin{verbatim}
## [1] 6.93
\end{verbatim}

What does the ``prob'' parameter change? When might you use different
values here?

\(~\)

Okay now lets build a toy data set

\begin{Shaded}
\begin{Highlighting}[]
\CommentTok{\# Set a random seed}
\FunctionTok{set.seed}\NormalTok{(}\DecValTok{103}\NormalTok{) }

\CommentTok{\# Define a data frame with our randomly generated data}
\NormalTok{randomData }\OtherTok{\textless{}{-}} \FunctionTok{data.frame}\NormalTok{(}\StringTok{\textquotesingle{}SubjectID\textquotesingle{}} \OtherTok{=} \FunctionTok{seq}\NormalTok{(}\DecValTok{1}\SpecialCharTok{:}\DecValTok{1000}\NormalTok{), }
                         \StringTok{\textquotesingle{}systolicBP\textquotesingle{}} \OtherTok{=} \FunctionTok{rnorm}\NormalTok{(}\AttributeTok{n =} \DecValTok{1000}\NormalTok{,}\AttributeTok{mean =} \DecValTok{128}\NormalTok{,}\AttributeTok{sd =} \DecValTok{20}\NormalTok{),}
                         \StringTok{\textquotesingle{}diastolicBP\textquotesingle{}} \OtherTok{=} \FunctionTok{rnorm}\NormalTok{(}\AttributeTok{n =} \DecValTok{1000}\NormalTok{,}\AttributeTok{mean =} \DecValTok{71}\NormalTok{,}\AttributeTok{sd =} \DecValTok{10}\NormalTok{),}
                         \StringTok{\textquotesingle{}Age\textquotesingle{}} \OtherTok{=} \FunctionTok{trunc}\NormalTok{(}\FunctionTok{runif}\NormalTok{(}\AttributeTok{n =} \DecValTok{1000}\NormalTok{,}\AttributeTok{min =} \DecValTok{18}\NormalTok{,}\AttributeTok{max =} \DecValTok{70}\NormalTok{)),}
                         \StringTok{\textquotesingle{}Male\textquotesingle{}} \OtherTok{=} \FunctionTok{rbinom}\NormalTok{(}\AttributeTok{n =} \DecValTok{1000}\NormalTok{,}\AttributeTok{size =} \DecValTok{1}\NormalTok{,}\AttributeTok{prob =} \FloatTok{0.5}\NormalTok{))}

\CommentTok{\# Take a peak at the top entries for our dataset }
\FunctionTok{head}\NormalTok{(randomData)}
\end{Highlighting}
\end{Shaded}

\begin{verbatim}
##   SubjectID systolicBP diastolicBP Age Male
## 1         1  112.28054    75.52894  52    0
## 2         2  129.09478    59.95778  56    0
## 3         3  104.54879    74.51568  25    1
## 4         4  124.65374    52.99577  41    0
## 5         5   90.69937    71.17388  41    0
## 6         6  125.59120    69.50961  31    0
\end{verbatim}

\begin{Shaded}
\begin{Highlighting}[]
\CommentTok{\# Assess the mean and standard deviation for systolic blood pressure}
\FunctionTok{mean}\NormalTok{(randomData}\SpecialCharTok{$}\NormalTok{systolicBP)}
\end{Highlighting}
\end{Shaded}

\begin{verbatim}
## [1] 128.4313
\end{verbatim}

\begin{Shaded}
\begin{Highlighting}[]
\FunctionTok{sd}\NormalTok{(randomData}\SpecialCharTok{$}\NormalTok{systolicBP)}
\end{Highlighting}
\end{Shaded}

\begin{verbatim}
## [1] 19.37998
\end{verbatim}

\begin{Shaded}
\begin{Highlighting}[]
\CommentTok{\# Visualize the distribution of diastolic blood pressure}
\FunctionTok{hist}\NormalTok{(randomData}\SpecialCharTok{$}\NormalTok{diastolicBP,}\AttributeTok{main =} \StringTok{\textquotesingle{}Randomly Generated Data\textquotesingle{}}\NormalTok{,}\AttributeTok{xlab =} \StringTok{\textquotesingle{}Diastolic Blood Pressure (mmHg)\textquotesingle{}}\NormalTok{)}
\end{Highlighting}
\end{Shaded}

\pandocbounded{\includegraphics[keepaspectratio]{DataWranglingI_StudentVersion_files/figure-latex/unnamed-chunk-5-1.pdf}}

\begin{Shaded}
\begin{Highlighting}[]
\CommentTok{\# Visualize the distribution of age}
\FunctionTok{hist}\NormalTok{(randomData}\SpecialCharTok{$}\NormalTok{Age,}\AttributeTok{main =} \StringTok{\textquotesingle{}Randomly Generated Data\textquotesingle{}}\NormalTok{,}\AttributeTok{xlab =} \StringTok{\textquotesingle{}Age (years)\textquotesingle{}}\NormalTok{)}
\end{Highlighting}
\end{Shaded}

\pandocbounded{\includegraphics[keepaspectratio]{DataWranglingI_StudentVersion_files/figure-latex/unnamed-chunk-5-2.pdf}}

What differences do you notice in the distribution of these two
variables?

\(~\)

\begin{Shaded}
\begin{Highlighting}[]
\CommentTok{\# Check the distribution of our variables}
\FunctionTok{summary}\NormalTok{(randomData)}
\end{Highlighting}
\end{Shaded}

\begin{verbatim}
##    SubjectID        systolicBP      diastolicBP         Age       
##  Min.   :   1.0   Min.   : 67.48   Min.   :43.07   Min.   :18.00  
##  1st Qu.: 250.8   1st Qu.:115.12   1st Qu.:64.15   1st Qu.:29.75  
##  Median : 500.5   Median :127.77   Median :70.67   Median :44.00  
##  Mean   : 500.5   Mean   :128.43   Mean   :70.99   Mean   :43.54  
##  3rd Qu.: 750.2   3rd Qu.:141.24   3rd Qu.:78.02   3rd Qu.:56.00  
##  Max.   :1000.0   Max.   :207.49   Max.   :98.99   Max.   :69.00  
##       Male      
##  Min.   :0.000  
##  1st Qu.:0.000  
##  Median :1.000  
##  Mean   :0.525  
##  3rd Qu.:1.000  
##  Max.   :1.000
\end{verbatim}

Is it summarizing the factor variable the way we want? Why or why not?

\(~\)

\subsubsection{Deriving New Variables}\label{deriving-new-variables}

Lets try converting the binary variable to a more logical format

\begin{Shaded}
\begin{Highlighting}[]
\CommentTok{\# Define a new factor variable from an old binary}
\NormalTok{randomData}\SpecialCharTok{$}\NormalTok{BiologicalSex }\OtherTok{\textless{}{-}} \FunctionTok{factor}\NormalTok{(}\FunctionTok{ifelse}\NormalTok{(randomData}\SpecialCharTok{$}\NormalTok{Male }\SpecialCharTok{==} \DecValTok{1}\NormalTok{,}\StringTok{\textquotesingle{}Male\textquotesingle{}}\NormalTok{,}\StringTok{\textquotesingle{}Female\textquotesingle{}}\NormalTok{))}

\CommentTok{\# Re{-}check the distribution of our variables}
\FunctionTok{summary}\NormalTok{(randomData)}
\end{Highlighting}
\end{Shaded}

\begin{verbatim}
##    SubjectID        systolicBP      diastolicBP         Age       
##  Min.   :   1.0   Min.   : 67.48   Min.   :43.07   Min.   :18.00  
##  1st Qu.: 250.8   1st Qu.:115.12   1st Qu.:64.15   1st Qu.:29.75  
##  Median : 500.5   Median :127.77   Median :70.67   Median :44.00  
##  Mean   : 500.5   Mean   :128.43   Mean   :70.99   Mean   :43.54  
##  3rd Qu.: 750.2   3rd Qu.:141.24   3rd Qu.:78.02   3rd Qu.:56.00  
##  Max.   :1000.0   Max.   :207.49   Max.   :98.99   Max.   :69.00  
##       Male       BiologicalSex
##  Min.   :0.000   Female:475   
##  1st Qu.:0.000   Male  :525   
##  Median :1.000                
##  Mean   :0.525                
##  3rd Qu.:1.000                
##  Max.   :1.000
\end{verbatim}

\(~\)

We can also generate a new binary variable from a continuous variable

\begin{Shaded}
\begin{Highlighting}[]
\CommentTok{\# Define variable specifying age above 65 (medicare eligible)}
\NormalTok{randomData}\SpecialCharTok{$}\NormalTok{MedicareAge }\OtherTok{\textless{}{-}} \FunctionTok{ifelse}\NormalTok{(randomData}\SpecialCharTok{$}\NormalTok{Age }\SpecialCharTok{\textless{}} \DecValTok{65}\NormalTok{,F,T)}

\CommentTok{\# Re{-}check the distribution of our variables}
\FunctionTok{summary}\NormalTok{(randomData)}
\end{Highlighting}
\end{Shaded}

\begin{verbatim}
##    SubjectID        systolicBP      diastolicBP         Age       
##  Min.   :   1.0   Min.   : 67.48   Min.   :43.07   Min.   :18.00  
##  1st Qu.: 250.8   1st Qu.:115.12   1st Qu.:64.15   1st Qu.:29.75  
##  Median : 500.5   Median :127.77   Median :70.67   Median :44.00  
##  Mean   : 500.5   Mean   :128.43   Mean   :70.99   Mean   :43.54  
##  3rd Qu.: 750.2   3rd Qu.:141.24   3rd Qu.:78.02   3rd Qu.:56.00  
##  Max.   :1000.0   Max.   :207.49   Max.   :98.99   Max.   :69.00  
##       Male       BiologicalSex MedicareAge    
##  Min.   :0.000   Female:475    Mode :logical  
##  1st Qu.:0.000   Male  :525    FALSE:903      
##  Median :1.000                 TRUE :97       
##  Mean   :0.525                                
##  3rd Qu.:1.000                                
##  Max.   :1.000
\end{verbatim}

\subsubsection{Subsetting Data}\label{subsetting-data}

As we learned in a previous lecture, we can subset a data frame in a few
ways

\begin{Shaded}
\begin{Highlighting}[]
\CommentTok{\# Subset to only those at medicare age}
\NormalTok{medicareData1 }\OtherTok{\textless{}{-}}\NormalTok{ randomData[}\FunctionTok{which}\NormalTok{(randomData}\SpecialCharTok{$}\NormalTok{MedicareAge }\SpecialCharTok{==}\NormalTok{ T),]}
\NormalTok{medicareData2 }\OtherTok{\textless{}{-}}\NormalTok{ randomData[}\FunctionTok{which}\NormalTok{(randomData}\SpecialCharTok{$}\NormalTok{Age }\SpecialCharTok{\textgreater{}=} \DecValTok{65}\NormalTok{),]}

\CommentTok{\# We can use the following statement to ensure both methods produced the same result}
\FunctionTok{all}\NormalTok{(medicareData1 }\SpecialCharTok{==}\NormalTok{ medicareData2)}
\end{Highlighting}
\end{Shaded}

\begin{verbatim}
## [1] TRUE
\end{verbatim}

\begin{Shaded}
\begin{Highlighting}[]
\FunctionTok{table}\NormalTok{(medicareData1 }\SpecialCharTok{==}\NormalTok{ medicareData2)}
\end{Highlighting}
\end{Shaded}

\begin{verbatim}
## 
## TRUE 
##  679
\end{verbatim}

\(~\)

Now we can look at the distribution in diastolic blood pressure just
among individuals meeting a given age cutoff

\begin{Shaded}
\begin{Highlighting}[]
\CommentTok{\# Visualize the distribution of diastolic blood pressure in Medicare Only}
\FunctionTok{hist}\NormalTok{(medicareData1}\SpecialCharTok{$}\NormalTok{diastolicBP,}\AttributeTok{main =} \StringTok{\textquotesingle{}Randomly Generated Data: Medicare Eligible Only\textquotesingle{}}\NormalTok{,}\AttributeTok{xlab =} \StringTok{\textquotesingle{}Diastolic Blood Pressure (mmHg)\textquotesingle{}}\NormalTok{)}
\end{Highlighting}
\end{Shaded}

\pandocbounded{\includegraphics[keepaspectratio]{DataWranglingI_StudentVersion_files/figure-latex/unnamed-chunk-10-1.pdf}}

\begin{Shaded}
\begin{Highlighting}[]
\CommentTok{\# Visualize the distribution of diastolic blood pressure in full dataset}
\FunctionTok{hist}\NormalTok{(randomData}\SpecialCharTok{$}\NormalTok{diastolicBP,}\AttributeTok{main =} \StringTok{\textquotesingle{}Randomly Generated Data: Full Cohort\textquotesingle{}}\NormalTok{,}\AttributeTok{xlab =} \StringTok{\textquotesingle{}Diastolic Blood Pressure (mmHg)\textquotesingle{}}\NormalTok{)}
\end{Highlighting}
\end{Shaded}

\pandocbounded{\includegraphics[keepaspectratio]{DataWranglingI_StudentVersion_files/figure-latex/unnamed-chunk-10-2.pdf}}

Does the distribution in the subsetted data look any different than in
the full data set? Should it?

\subsubsection{\texorpdfstring{Basic Plotting Using
\emph{ggpubr}}{Basic Plotting Using ggpubr}}\label{basic-plotting-using-ggpubr}

\begin{Shaded}
\begin{Highlighting}[]
\CommentTok{\# Run the following line of code one time to install the package if you haven\textquotesingle{}t already}
\CommentTok{\# install.packages(\textquotesingle{}ggpubr\textquotesingle{})}

\CommentTok{\# Generate a scatter plot of age by systolic blood pressure}
\NormalTok{ggpubr}\SpecialCharTok{::}\FunctionTok{ggscatter}\NormalTok{(randomData,}\AttributeTok{x =} \StringTok{\textquotesingle{}Age\textquotesingle{}}\NormalTok{,}\AttributeTok{y =} \StringTok{\textquotesingle{}systolicBP\textquotesingle{}}\NormalTok{,}\AttributeTok{ylab =} \StringTok{\textquotesingle{}Systolic Blood Pressure (mmHg)\textquotesingle{}}\NormalTok{)}
\end{Highlighting}
\end{Shaded}

\pandocbounded{\includegraphics[keepaspectratio]{DataWranglingI_StudentVersion_files/figure-latex/unnamed-chunk-11-1.pdf}}

\begin{Shaded}
\begin{Highlighting}[]
\CommentTok{\# Generate a boxplot for diastolic blood pressure distrubition by biological sex in our original dataset}
\NormalTok{ggpubr}\SpecialCharTok{::}\FunctionTok{ggboxplot}\NormalTok{(randomData,}\AttributeTok{x =} \StringTok{\textquotesingle{}BiologicalSex\textquotesingle{}}\NormalTok{,}\AttributeTok{y =} \StringTok{\textquotesingle{}systolicBP\textquotesingle{}}\NormalTok{,}\AttributeTok{xlab =} \StringTok{\textquotesingle{}Biological Sex\textquotesingle{}}\NormalTok{,}
                  \AttributeTok{ylab =} \StringTok{\textquotesingle{}Systolic Blood Pressure (mmHg)\textquotesingle{}}\NormalTok{,}\AttributeTok{add =} \StringTok{\textquotesingle{}jitter\textquotesingle{}}\NormalTok{)}
\end{Highlighting}
\end{Shaded}

\pandocbounded{\includegraphics[keepaspectratio]{DataWranglingI_StudentVersion_files/figure-latex/unnamed-chunk-11-2.pdf}}

Was there any visual association between age or sex and blood pressure?

Would we expect there to be based on how we generated the data?

\(~\)

We can also use \textbf{ggarrange} to organize plots in a grid.

\begin{Shaded}
\begin{Highlighting}[]
\NormalTok{ggpubr}\SpecialCharTok{::}\FunctionTok{ggarrange}\NormalTok{(}
\NormalTok{   ggpubr}\SpecialCharTok{::}\FunctionTok{ggscatter}\NormalTok{(randomData,}\AttributeTok{x =} \StringTok{\textquotesingle{}Age\textquotesingle{}}\NormalTok{,}\AttributeTok{y =} \StringTok{\textquotesingle{}systolicBP\textquotesingle{}}\NormalTok{,}
                     \AttributeTok{xlab =} \StringTok{\textquotesingle{}Age (yrs)\textquotesingle{}}\NormalTok{,}\AttributeTok{ylab =} \StringTok{\textquotesingle{}Systolic Blood Pressure (mmHg)\textquotesingle{}}\NormalTok{),}
\NormalTok{   ggpubr}\SpecialCharTok{::}\FunctionTok{ggboxplot}\NormalTok{(randomData,}\AttributeTok{x =} \StringTok{\textquotesingle{}BiologicalSex\textquotesingle{}}\NormalTok{,}\AttributeTok{y =} \StringTok{\textquotesingle{}systolicBP\textquotesingle{}}\NormalTok{,}
                     \AttributeTok{xlab =} \StringTok{\textquotesingle{}Biological Sex\textquotesingle{}}\NormalTok{,}\AttributeTok{ylab =} \StringTok{\textquotesingle{}Systolic Blood Pressure (mmHg)\textquotesingle{}}\NormalTok{),}
   \AttributeTok{ncol =} \DecValTok{2}\NormalTok{,}\AttributeTok{nrow =} \DecValTok{1}\CommentTok{\#,labels = \textquotesingle{}auto\textquotesingle{}}
\NormalTok{)}
\end{Highlighting}
\end{Shaded}

\pandocbounded{\includegraphics[keepaspectratio]{DataWranglingI_StudentVersion_files/figure-latex/unnamed-chunk-12-1.pdf}}

\(~\)

What if we want to look at systolic and diastolic blood pressure in the
same plot? To do this, we need to convert out data to a long format.

\subsubsection{Wide vs Long Dataframes}\label{wide-vs-long-dataframes}

\textbf{Wide} format data will have each subject in a single row and
then multiple measures related to that subject in a unique column.

\begin{Shaded}
\begin{Highlighting}[]
\CommentTok{\# Wide format data}
\FunctionTok{data.frame}\NormalTok{(}\StringTok{\textquotesingle{}PatientID\textquotesingle{}} \OtherTok{=} \FunctionTok{c}\NormalTok{(}\StringTok{\textquotesingle{}P001\textquotesingle{}}\NormalTok{,}\StringTok{\textquotesingle{}P002\textquotesingle{}}\NormalTok{,}\StringTok{\textquotesingle{}P003\textquotesingle{}}\NormalTok{),}
                          \StringTok{\textquotesingle{}Age\textquotesingle{}} \OtherTok{=} \FunctionTok{c}\NormalTok{(}\DecValTok{24}\NormalTok{,}\DecValTok{35}\NormalTok{,}\DecValTok{27}\NormalTok{),}
                          \StringTok{\textquotesingle{}Height.in\textquotesingle{}} \OtherTok{=} \FunctionTok{c}\NormalTok{(}\DecValTok{67}\NormalTok{,}\DecValTok{70}\NormalTok{,}\DecValTok{64}\NormalTok{))}
\end{Highlighting}
\end{Shaded}

\begin{verbatim}
##   PatientID Age Height.in
## 1      P001  24        67
## 2      P002  35        70
## 3      P003  27        64
\end{verbatim}

\textbf{Long} format data will have each measurement in a new row.

\begin{Shaded}
\begin{Highlighting}[]
\CommentTok{\# Long format data}
\FunctionTok{data.frame}\NormalTok{(}\StringTok{\textquotesingle{}PatientID\textquotesingle{}} \OtherTok{=} \FunctionTok{c}\NormalTok{(}\StringTok{\textquotesingle{}P001\textquotesingle{}}\NormalTok{,}\StringTok{\textquotesingle{}P001\textquotesingle{}}\NormalTok{,}\StringTok{\textquotesingle{}P002\textquotesingle{}}\NormalTok{,}\StringTok{\textquotesingle{}P002\textquotesingle{}}\NormalTok{,}\StringTok{\textquotesingle{}P003\textquotesingle{}}\NormalTok{,}\StringTok{\textquotesingle{}P003\textquotesingle{}}\NormalTok{),}
           \StringTok{\textquotesingle{}Measure\textquotesingle{}} \OtherTok{=} \FunctionTok{c}\NormalTok{(}\StringTok{\textquotesingle{}Age\textquotesingle{}}\NormalTok{,}\StringTok{\textquotesingle{}Height.in\textquotesingle{}}\NormalTok{,}\StringTok{\textquotesingle{}Age\textquotesingle{}}\NormalTok{,}\StringTok{\textquotesingle{}Height.in\textquotesingle{}}\NormalTok{,}\StringTok{\textquotesingle{}Age\textquotesingle{}}\NormalTok{,}\StringTok{\textquotesingle{}Height.in\textquotesingle{}}\NormalTok{),}
           \StringTok{\textquotesingle{}Value\textquotesingle{}} \OtherTok{=} \FunctionTok{c}\NormalTok{(}\DecValTok{24}\NormalTok{,}\DecValTok{67}\NormalTok{,}\DecValTok{35}\NormalTok{,}\DecValTok{70}\NormalTok{,}\DecValTok{27}\NormalTok{,}\DecValTok{64}\NormalTok{))}
\end{Highlighting}
\end{Shaded}

\begin{verbatim}
##   PatientID   Measure Value
## 1      P001       Age    24
## 2      P001 Height.in    67
## 3      P002       Age    35
## 4      P002 Height.in    70
## 5      P003       Age    27
## 6      P003 Height.in    64
\end{verbatim}

While this might seem odd for this type of use case, it is a very useful
tool for data sets with \textbf{repeated measures}, i.e.~measurements
taken over time like this:

\begin{Shaded}
\begin{Highlighting}[]
\CommentTok{\# Wide format repeated measures}
\FunctionTok{data.frame}\NormalTok{(}\StringTok{\textquotesingle{}PatientID\textquotesingle{}} \OtherTok{=} \FunctionTok{c}\NormalTok{(}\StringTok{\textquotesingle{}P001\textquotesingle{}}\NormalTok{,}\StringTok{\textquotesingle{}P002\textquotesingle{}}\NormalTok{,}\StringTok{\textquotesingle{}P003\textquotesingle{}}\NormalTok{),}
           \StringTok{\textquotesingle{}Height.in.5y\textquotesingle{}} \OtherTok{=} \FunctionTok{c}\NormalTok{(}\DecValTok{40}\NormalTok{,}\DecValTok{43}\NormalTok{,}\DecValTok{39}\NormalTok{),}
           \StringTok{\textquotesingle{}Height.in.10y\textquotesingle{}} \OtherTok{=} \FunctionTok{c}\NormalTok{(}\DecValTok{53}\NormalTok{,}\DecValTok{58}\NormalTok{,}\DecValTok{51}\NormalTok{))}
\end{Highlighting}
\end{Shaded}

\begin{verbatim}
##   PatientID Height.in.5y Height.in.10y
## 1      P001           40            53
## 2      P002           43            58
## 3      P003           39            51
\end{verbatim}

\begin{Shaded}
\begin{Highlighting}[]
\CommentTok{\# Long format repeated measures}
\FunctionTok{data.frame}\NormalTok{(}\StringTok{\textquotesingle{}PatientID\textquotesingle{}} \OtherTok{=} \FunctionTok{c}\NormalTok{(}\StringTok{\textquotesingle{}P001\textquotesingle{}}\NormalTok{,}\StringTok{\textquotesingle{}P001\textquotesingle{}}\NormalTok{,}\StringTok{\textquotesingle{}P002\textquotesingle{}}\NormalTok{,}\StringTok{\textquotesingle{}P002\textquotesingle{}}\NormalTok{,}\StringTok{\textquotesingle{}P003\textquotesingle{}}\NormalTok{,}\StringTok{\textquotesingle{}P003\textquotesingle{}}\NormalTok{),}
           \StringTok{\textquotesingle{}Age.yrs\textquotesingle{}} \OtherTok{=} \FunctionTok{c}\NormalTok{(}\DecValTok{5}\NormalTok{,}\DecValTok{10}\NormalTok{,}\DecValTok{5}\NormalTok{,}\DecValTok{10}\NormalTok{,}\DecValTok{5}\NormalTok{,}\DecValTok{10}\NormalTok{),}
           \StringTok{\textquotesingle{}Height.in\textquotesingle{}} \OtherTok{=} \FunctionTok{c}\NormalTok{(}\DecValTok{40}\NormalTok{,}\DecValTok{53}\NormalTok{,}\DecValTok{43}\NormalTok{,}\DecValTok{58}\NormalTok{,}\DecValTok{39}\NormalTok{,}\DecValTok{51}\NormalTok{))}
\end{Highlighting}
\end{Shaded}

\begin{verbatim}
##   PatientID Age.yrs Height.in
## 1      P001       5        40
## 2      P001      10        53
## 3      P002       5        43
## 4      P002      10        58
## 5      P003       5        39
## 6      P003      10        51
\end{verbatim}

\subsubsection{Long Format Data}\label{long-format-data}

Currently, our data is in a \textbf{wide} format because each subject is
only in a single row and all measures from that subject are in columns.
There are a few different ways to convert between wide an long format
data.

To achieve this today, we will use the \emph{melt} function from the
package \emph{reshape2}. Next class, we will discuss how to achieve this
in \emph{tidyverse}.

\begin{Shaded}
\begin{Highlighting}[]
\CommentTok{\# Run the following line of code one time to install the package if you haven\textquotesingle{}t already}
\CommentTok{\# install.packages(\textquotesingle{}reshape2\textquotesingle{})}

\CommentTok{\# Melt the data frame into a long form}
\NormalTok{longData }\OtherTok{\textless{}{-}}\NormalTok{ reshape2}\SpecialCharTok{::}\FunctionTok{melt}\NormalTok{(randomData[,}\FunctionTok{c}\NormalTok{(}\StringTok{\textquotesingle{}SubjectID\textquotesingle{}}\NormalTok{,}\StringTok{\textquotesingle{}systolicBP\textquotesingle{}}\NormalTok{,}\StringTok{\textquotesingle{}diastolicBP\textquotesingle{}}\NormalTok{,}\StringTok{\textquotesingle{}Age\textquotesingle{}}\NormalTok{,}\StringTok{\textquotesingle{}BiologicalSex\textquotesingle{}}\NormalTok{)],}
                           \AttributeTok{id.vars =} \FunctionTok{c}\NormalTok{(}\StringTok{\textquotesingle{}SubjectID\textquotesingle{}}\NormalTok{,}\StringTok{\textquotesingle{}Age\textquotesingle{}}\NormalTok{,}\StringTok{\textquotesingle{}BiologicalSex\textquotesingle{}}\NormalTok{),}\AttributeTok{value.name =} \StringTok{\textquotesingle{}BP\textquotesingle{}}\NormalTok{,}
                           \AttributeTok{variable.name =} \StringTok{\textquotesingle{}BP.Type\textquotesingle{}}\NormalTok{)}
\CommentTok{\# Check enteries for our first subject}
\NormalTok{longData[longData}\SpecialCharTok{$}\NormalTok{SubjectID }\SpecialCharTok{==} \DecValTok{1}\NormalTok{,]}
\end{Highlighting}
\end{Shaded}

\begin{verbatim}
##      SubjectID Age BiologicalSex     BP.Type        BP
## 1            1  52        Female  systolicBP 112.28054
## 1001         1  52        Female diastolicBP  75.52894
\end{verbatim}

\begin{Shaded}
\begin{Highlighting}[]
\CommentTok{\# How it would look if we forgot one of our ID variables}
\NormalTok{longData2 }\OtherTok{\textless{}{-}}\NormalTok{ reshape2}\SpecialCharTok{::}\FunctionTok{melt}\NormalTok{(randomData[,}\FunctionTok{c}\NormalTok{(}\StringTok{\textquotesingle{}SubjectID\textquotesingle{}}\NormalTok{,}\StringTok{\textquotesingle{}systolicBP\textquotesingle{}}\NormalTok{,}\StringTok{\textquotesingle{}diastolicBP\textquotesingle{}}\NormalTok{,}\StringTok{\textquotesingle{}Age\textquotesingle{}}\NormalTok{,}\StringTok{\textquotesingle{}BiologicalSex\textquotesingle{}}\NormalTok{)],}
                           \AttributeTok{id.vars =} \FunctionTok{c}\NormalTok{(}\StringTok{\textquotesingle{}SubjectID\textquotesingle{}}\NormalTok{,}\StringTok{\textquotesingle{}Age\textquotesingle{}}\NormalTok{),}\AttributeTok{value.name =} \StringTok{\textquotesingle{}BP\textquotesingle{}}\NormalTok{,}
                           \AttributeTok{variable.name =} \StringTok{\textquotesingle{}BP.Type\textquotesingle{}}\NormalTok{)}
\end{Highlighting}
\end{Shaded}

\begin{verbatim}
## Warning: attributes are not identical across measure variables; they will be
## dropped
\end{verbatim}

\begin{Shaded}
\begin{Highlighting}[]
\FunctionTok{typeof}\NormalTok{(longData2}\SpecialCharTok{$}\NormalTok{BP)}
\end{Highlighting}
\end{Shaded}

\begin{verbatim}
## [1] "character"
\end{verbatim}

\begin{Shaded}
\begin{Highlighting}[]
\NormalTok{longData2[longData2}\SpecialCharTok{$}\NormalTok{SubjectID }\SpecialCharTok{==} \DecValTok{1}\NormalTok{,]}
\end{Highlighting}
\end{Shaded}

\begin{verbatim}
##      SubjectID Age       BP.Type               BP
## 1            1  52    systolicBP 112.280536472035
## 1001         1  52   diastolicBP 75.5289415224527
## 2001         1  52 BiologicalSex           Female
\end{verbatim}

\(~\)

We can also use the \emph{order} function to ensure that all our
subjects values are listed in sequence in the table or to look at
subjects who have the oldest age easily

\begin{Shaded}
\begin{Highlighting}[]
\CommentTok{\# Print first 20 values}
\NormalTok{longData}\SpecialCharTok{$}\NormalTok{Age[}\DecValTok{1}\SpecialCharTok{:}\DecValTok{20}\NormalTok{]}
\end{Highlighting}
\end{Shaded}

\begin{verbatim}
##  [1] 52 56 25 41 41 31 69 67 46 60 36 56 40 39 38 52 52 60 20 61
\end{verbatim}

\begin{Shaded}
\begin{Highlighting}[]
\FunctionTok{order}\NormalTok{(longData}\SpecialCharTok{$}\NormalTok{Age)[}\DecValTok{1}\SpecialCharTok{:}\DecValTok{20}\NormalTok{]}
\end{Highlighting}
\end{Shaded}

\begin{verbatim}
##  [1]   35  121  158  214  237  337  342  350  594  616  653  656  717  736  739
## [16]  846  976  991 1035 1121
\end{verbatim}

What does the \emph{order} function return?

\begin{Shaded}
\begin{Highlighting}[]
\NormalTok{longData}\SpecialCharTok{$}\NormalTok{Age[}\DecValTok{14}\NormalTok{]}
\end{Highlighting}
\end{Shaded}

\begin{verbatim}
## [1] 39
\end{verbatim}

\begin{Shaded}
\begin{Highlighting}[]
\NormalTok{longData[}\FunctionTok{order}\NormalTok{(longData}\SpecialCharTok{$}\NormalTok{Age),}\StringTok{\textquotesingle{}Age\textquotesingle{}}\NormalTok{][}\DecValTok{1}\SpecialCharTok{:}\DecValTok{20}\NormalTok{]}
\end{Highlighting}
\end{Shaded}

\begin{verbatim}
##  [1] 18 18 18 18 18 18 18 18 18 18 18 18 18 18 18 18 18 18 18 18
\end{verbatim}

\(~\)

\begin{Shaded}
\begin{Highlighting}[]
\CommentTok{\# Order by subject ID}
\FunctionTok{head}\NormalTok{(longData[}\FunctionTok{order}\NormalTok{(longData}\SpecialCharTok{$}\NormalTok{SubjectID,}\AttributeTok{decreasing =}\NormalTok{ F),])}
\end{Highlighting}
\end{Shaded}

\begin{verbatim}
##      SubjectID Age BiologicalSex     BP.Type        BP
## 1            1  52        Female  systolicBP 112.28054
## 1001         1  52        Female diastolicBP  75.52894
## 2            2  56        Female  systolicBP 129.09478
## 1002         2  56        Female diastolicBP  59.95778
## 3            3  25          Male  systolicBP 104.54879
## 1003         3  25          Male diastolicBP  74.51568
\end{verbatim}

\begin{Shaded}
\begin{Highlighting}[]
\CommentTok{\# Order by decreasing age}
\FunctionTok{head}\NormalTok{(longData[}\FunctionTok{order}\NormalTok{(longData}\SpecialCharTok{$}\NormalTok{Age,}\AttributeTok{decreasing =}\NormalTok{ T),])}
\end{Highlighting}
\end{Shaded}

\begin{verbatim}
##     SubjectID Age BiologicalSex    BP.Type       BP
## 7           7  69        Female systolicBP 144.5196
## 114       114  69          Male systolicBP 119.9431
## 124       124  69        Female systolicBP 140.5145
## 336       336  69          Male systolicBP 115.8081
## 363       363  69        Female systolicBP 139.2851
## 391       391  69          Male systolicBP 119.4529
\end{verbatim}

\(~\) The order function can be particularly helpful when trying to
merge two data frames and want to first ensure that subjects or samples
are in the same order in each data frame.

\subsubsection{Manipulating Character
Strings}\label{manipulating-character-strings}

If we don't like the end of systolic and diastolic labels having ``BP''
we can remove it using the \emph{sub} function.

\begin{Shaded}
\begin{Highlighting}[]
\CommentTok{\# Look for cases of "BP" and replace with ""}
\NormalTok{longData}\SpecialCharTok{$}\NormalTok{BP.Type }\OtherTok{\textless{}{-}} \FunctionTok{sub}\NormalTok{(}\AttributeTok{pattern =} \StringTok{\textquotesingle{}BP\textquotesingle{}}\NormalTok{,}\AttributeTok{replacement =} \StringTok{\textquotesingle{}\textquotesingle{}}\NormalTok{,}\AttributeTok{x =}\NormalTok{ longData}\SpecialCharTok{$}\NormalTok{BP.Type)}

\CommentTok{\# Now lets see what values we have for this variable}
\FunctionTok{table}\NormalTok{(longData}\SpecialCharTok{$}\NormalTok{BP.Type)}
\end{Highlighting}
\end{Shaded}

\begin{verbatim}
## 
## diastolic  systolic 
##      1000      1000
\end{verbatim}

\begin{Shaded}
\begin{Highlighting}[]
\CommentTok{\# We could also just replace the values using indexing as follows}
\NormalTok{longData[}\FunctionTok{which}\NormalTok{(longData}\SpecialCharTok{$}\NormalTok{BP.Type }\SpecialCharTok{==} \StringTok{\textquotesingle{}systolic\textquotesingle{}}\NormalTok{),}\StringTok{\textquotesingle{}BP.Type\textquotesingle{}}\NormalTok{] }\OtherTok{\textless{}{-}} \StringTok{\textquotesingle{}Systolic\textquotesingle{}}
\NormalTok{longData[}\FunctionTok{which}\NormalTok{(longData}\SpecialCharTok{$}\NormalTok{BP.Type }\SpecialCharTok{==} \StringTok{\textquotesingle{}diastolic\textquotesingle{}}\NormalTok{),}\StringTok{\textquotesingle{}BP.Type\textquotesingle{}}\NormalTok{] }\OtherTok{\textless{}{-}} \StringTok{\textquotesingle{}Diastolic\textquotesingle{}}

\CommentTok{\# Now lets check again what values we have for this variable}
\FunctionTok{table}\NormalTok{(longData}\SpecialCharTok{$}\NormalTok{BP.Type)}
\end{Highlighting}
\end{Shaded}

\begin{verbatim}
## 
## Diastolic  Systolic 
##      1000      1000
\end{verbatim}

\emph{sub} and \emph{gsub} can both be used to search and replace in
character strings. Can you identify the difference between what each one
does?

\begin{Shaded}
\begin{Highlighting}[]
\CommentTok{\# Define a list of character strings}
\NormalTok{charStrings }\OtherTok{\textless{}{-}} \FunctionTok{c}\NormalTok{(}\StringTok{\textquotesingle{}QBS Graduate Program at Dartmouth College\textquotesingle{}}\NormalTok{,}\StringTok{\textquotesingle{}QBS 103\textquotesingle{}}\NormalTok{,}\StringTok{\textquotesingle{}QBS! QBS! QBS!\textquotesingle{}}\NormalTok{)}

\CommentTok{\# sub function}
\FunctionTok{sub}\NormalTok{(}\AttributeTok{pattern =} \StringTok{\textquotesingle{}QBS\textquotesingle{}}\NormalTok{,}\AttributeTok{replacement =} \StringTok{\textquotesingle{}Dartmouth\textquotesingle{}}\NormalTok{,}\AttributeTok{x =}\NormalTok{ charStrings)}
\end{Highlighting}
\end{Shaded}

\begin{verbatim}
## [1] "Dartmouth Graduate Program at Dartmouth College"
## [2] "Dartmouth 103"                                  
## [3] "Dartmouth! QBS! QBS!"
\end{verbatim}

\begin{Shaded}
\begin{Highlighting}[]
\CommentTok{\# gsub function}
\FunctionTok{gsub}\NormalTok{(}\AttributeTok{pattern =} \StringTok{\textquotesingle{}QBS\textquotesingle{}}\NormalTok{,}\AttributeTok{replacement =} \StringTok{\textquotesingle{}Dartmouth\textquotesingle{}}\NormalTok{,}\AttributeTok{x =}\NormalTok{ charStrings)}
\end{Highlighting}
\end{Shaded}

\begin{verbatim}
## [1] "Dartmouth Graduate Program at Dartmouth College"
## [2] "Dartmouth 103"                                  
## [3] "Dartmouth! Dartmouth! Dartmouth!"
\end{verbatim}

\(~\)

This may seem easy enough to do on your own, but sometimes you will have
large lists of character strings between two dataframes that have been
entered in different formats.

Here, I have 2 lists of sample IDs that I need to make sure are in the
same order in both lists but you can see, they are formatted differently
across both lists.

\begin{Shaded}
\begin{Highlighting}[]
\CommentTok{\# Define two lists of sample names}
\NormalTok{d1 }\OtherTok{\textless{}{-}} \FunctionTok{c}\NormalTok{(}\StringTok{\textquotesingle{}20210323\_PB4\_01\_09.RCC\textquotesingle{}}\NormalTok{,}\StringTok{\textquotesingle{}20210323\_PB4\_01\_11.RCC\textquotesingle{}}\NormalTok{,}\StringTok{\textquotesingle{}20210323\_PB4\_01\_12.RCC\textquotesingle{}}\NormalTok{,}\StringTok{\textquotesingle{}20210401\_ch5{-}040121\_01\_01.RCC\textquotesingle{}}\NormalTok{,}\StringTok{\textquotesingle{}20210401\_ch5{-}040121\_01\_02.RCC\textquotesingle{}}\NormalTok{,}\StringTok{\textquotesingle{}20210401\_ch5{-}040121\_01\_02.RCC\textquotesingle{}}\NormalTok{,}\StringTok{\textquotesingle{}20210401\_ch5{-}040121\_01\_03.RCC\textquotesingle{}}\NormalTok{)}
\NormalTok{d2 }\OtherTok{\textless{}{-}} \FunctionTok{c}\NormalTok{(}\StringTok{\textquotesingle{}20210323{-}PB4\_01\_11.RCC\textquotesingle{}}\NormalTok{,}\StringTok{\textquotesingle{}20210323{-}PB4\_01\_12.RCC\textquotesingle{}}\NormalTok{,}\StringTok{\textquotesingle{}20210323{-}PB4\_01\_09.RCC\textquotesingle{}}\NormalTok{,}\StringTok{\textquotesingle{}20210401{-}ch5{-}040121{-}01{-}01.RCC\textquotesingle{}}\NormalTok{,}\StringTok{\textquotesingle{}20210401{-}ch5{-}040121{-}01{-}02.RCC\textquotesingle{}}\NormalTok{,}\StringTok{\textquotesingle{}20210401{-}ch5{-}040121{-}01{-}02.RCC\textquotesingle{}}\NormalTok{,}\StringTok{\textquotesingle{}20210401{-}ch5{-}040121{-}01{-}03.RCC\textquotesingle{}}\NormalTok{)}

\CommentTok{\# Check if same samples are in both lists}
\FunctionTok{table}\NormalTok{(d1 }\SpecialCharTok{\%in\%}\NormalTok{ d2)}
\end{Highlighting}
\end{Shaded}

\begin{verbatim}
## 
## FALSE 
##     7
\end{verbatim}

\begin{Shaded}
\begin{Highlighting}[]
\CommentTok{\# Compare formatting }
\FunctionTok{head}\NormalTok{(d1)}
\end{Highlighting}
\end{Shaded}

\begin{verbatim}
## [1] "20210323_PB4_01_09.RCC"        "20210323_PB4_01_11.RCC"       
## [3] "20210323_PB4_01_12.RCC"        "20210401_ch5-040121_01_01.RCC"
## [5] "20210401_ch5-040121_01_02.RCC" "20210401_ch5-040121_01_02.RCC"
\end{verbatim}

\begin{Shaded}
\begin{Highlighting}[]
\FunctionTok{head}\NormalTok{(d2)}
\end{Highlighting}
\end{Shaded}

\begin{verbatim}
## [1] "20210323-PB4_01_11.RCC"        "20210323-PB4_01_12.RCC"       
## [3] "20210323-PB4_01_09.RCC"        "20210401-ch5-040121-01-01.RCC"
## [5] "20210401-ch5-040121-01-02.RCC" "20210401-ch5-040121-01-02.RCC"
\end{verbatim}

\(~\) We can change the formatting using \emph{gsub}

\begin{Shaded}
\begin{Highlighting}[]
\CommentTok{\# Replace all "\_" with "{-}" in list of sample names}
\NormalTok{d1 }\OtherTok{\textless{}{-}} \FunctionTok{gsub}\NormalTok{(d1,}\AttributeTok{pattern =} \StringTok{\textquotesingle{}\_\textquotesingle{}}\NormalTok{,}\AttributeTok{replacement =} \StringTok{\textquotesingle{}{-}\textquotesingle{}}\NormalTok{)}
\NormalTok{d2 }\OtherTok{\textless{}{-}} \FunctionTok{gsub}\NormalTok{(d2,}\AttributeTok{pattern =} \StringTok{\textquotesingle{}\_\textquotesingle{}}\NormalTok{,}\AttributeTok{replacement =} \StringTok{\textquotesingle{}{-}\textquotesingle{}}\NormalTok{)}

\CommentTok{\# Verify all names in d1 are now also in d2}
\FunctionTok{table}\NormalTok{(d1 }\SpecialCharTok{\%in\%}\NormalTok{ d2)}
\end{Highlighting}
\end{Shaded}

\begin{verbatim}
## 
## TRUE 
##    7
\end{verbatim}

\begin{Shaded}
\begin{Highlighting}[]
\CommentTok{\# Check if D1 and D2 are in the same order}
\FunctionTok{table}\NormalTok{(d1 }\SpecialCharTok{==}\NormalTok{ d2)}
\end{Highlighting}
\end{Shaded}

\begin{verbatim}
## 
## FALSE  TRUE 
##     3     4
\end{verbatim}

\begin{Shaded}
\begin{Highlighting}[]
\CommentTok{\# Reorder both vectors}
\NormalTok{d1 }\OtherTok{\textless{}{-}}\NormalTok{ d1[}\FunctionTok{order}\NormalTok{(d1)]}
\NormalTok{d2 }\OtherTok{\textless{}{-}}\NormalTok{ d2[}\FunctionTok{order}\NormalTok{(d2)]}

\CommentTok{\# Check if D1 and D2 are in the same order}
\FunctionTok{table}\NormalTok{(d1 }\SpecialCharTok{==}\NormalTok{ d2)}
\end{Highlighting}
\end{Shaded}

\begin{verbatim}
## 
## TRUE 
##    7
\end{verbatim}

\subsubsection{\texorpdfstring{Adding Color to Our \emph{ggpubr}
Plots}{Adding Color to Our ggpubr Plots}}\label{adding-color-to-our-ggpubr-plots}

Now we can go back and generate out plots with both measures of blood
pressure in one plot.

\begin{Shaded}
\begin{Highlighting}[]
\CommentTok{\# Generate a scatter plot of age by systolic blood pressure}
\NormalTok{ggpubr}\SpecialCharTok{::}\FunctionTok{ggscatter}\NormalTok{(longData,}\AttributeTok{x =} \StringTok{\textquotesingle{}Age\textquotesingle{}}\NormalTok{,}\AttributeTok{y =} \StringTok{\textquotesingle{}BP\textquotesingle{}}\NormalTok{,}\AttributeTok{color =} \StringTok{\textquotesingle{}BP.Type\textquotesingle{}}\NormalTok{)}
\end{Highlighting}
\end{Shaded}

\pandocbounded{\includegraphics[keepaspectratio]{DataWranglingI_StudentVersion_files/figure-latex/unnamed-chunk-25-1.pdf}}

\begin{Shaded}
\begin{Highlighting}[]
\CommentTok{\# Generate a boxplot for diastolic blood pressure distrubition by biological sex in our original dataset}
\NormalTok{ggpubr}\SpecialCharTok{::}\FunctionTok{ggboxplot}\NormalTok{(longData,}\AttributeTok{x =} \StringTok{\textquotesingle{}BP.Type\textquotesingle{}}\NormalTok{,}\AttributeTok{y =} \StringTok{\textquotesingle{}BP\textquotesingle{}}\NormalTok{,}\AttributeTok{color =} \StringTok{\textquotesingle{}BiologicalSex\textquotesingle{}}\NormalTok{,}
                  \AttributeTok{ylab =} \StringTok{\textquotesingle{}Blood Pressure (mmHg)\textquotesingle{}}\NormalTok{,}\AttributeTok{xlab =} \StringTok{\textquotesingle{}\textquotesingle{}}\NormalTok{)}
\end{Highlighting}
\end{Shaded}

\pandocbounded{\includegraphics[keepaspectratio]{DataWranglingI_StudentVersion_files/figure-latex/unnamed-chunk-25-2.pdf}}

\subsubsection{Converting Back to Wide Format
Data}\label{converting-back-to-wide-format-data}

We can also change our data back into a \textbf{wide} format using the
\emph{cast} function as follows:

\begin{Shaded}
\begin{Highlighting}[]
\CommentTok{\# Cast into a data frame}
\NormalTok{wideData }\OtherTok{\textless{}{-}}\NormalTok{ reshape2}\SpecialCharTok{::}\FunctionTok{dcast}\NormalTok{(longData,}\AttributeTok{formula =}\NormalTok{  SubjectID }\SpecialCharTok{+}\NormalTok{ Age }\SpecialCharTok{+}\NormalTok{ BiologicalSex }\SpecialCharTok{\textasciitilde{}}\NormalTok{ BP.Type,}\AttributeTok{value.var =} \StringTok{"BP"}\NormalTok{)}
\CommentTok{\# Note: using dcast because we want a dataframe as output. }
\CommentTok{\# If we want a vector or matrix as output, we use acast}

\CommentTok{\# Lets see if our data look the same as when we started}
\FunctionTok{head}\NormalTok{(wideData)}
\end{Highlighting}
\end{Shaded}

\begin{verbatim}
##   SubjectID Age BiologicalSex Diastolic  Systolic
## 1         1  52        Female  75.52894 112.28054
## 2         2  56        Female  59.95778 129.09478
## 3         3  25          Male  74.51568 104.54879
## 4         4  41        Female  52.99577 124.65374
## 5         5  41        Female  71.17388  90.69937
## 6         6  31        Female  69.50961 125.59120
\end{verbatim}

\begin{Shaded}
\begin{Highlighting}[]
\CommentTok{\# Our original data set}
\FunctionTok{head}\NormalTok{(randomData)}
\end{Highlighting}
\end{Shaded}

\begin{verbatim}
##   SubjectID systolicBP diastolicBP Age Male BiologicalSex MedicareAge
## 1         1  112.28054    75.52894  52    0        Female       FALSE
## 2         2  129.09478    59.95778  56    0        Female       FALSE
## 3         3  104.54879    74.51568  25    1          Male       FALSE
## 4         4  124.65374    52.99577  41    0        Female       FALSE
## 5         5   90.69937    71.17388  41    0        Female       FALSE
## 6         6  125.59120    69.50961  31    0        Female       FALSE
\end{verbatim}

We can also use the \emph{dcast} function to make summary values. For
example, what if we wanted to known the average BP for males and females
based on their age group.

\begin{Shaded}
\begin{Highlighting}[]
\CommentTok{\# Generate table (note here we drop subject ID from the formula)}
\NormalTok{summaryTable }\OtherTok{\textless{}{-}}\NormalTok{ reshape2}\SpecialCharTok{::}\FunctionTok{dcast}\NormalTok{(longData,}\AttributeTok{formula =}\NormalTok{ Age }\SpecialCharTok{+}\NormalTok{ BiologicalSex }\SpecialCharTok{\textasciitilde{}}\NormalTok{ BP.Type,}
                                \AttributeTok{fun.aggregate =}\NormalTok{ mean,}\AttributeTok{value.var =} \StringTok{\textquotesingle{}BP\textquotesingle{}}\NormalTok{)}

\CommentTok{\# Check the dimmensions }
\FunctionTok{dim}\NormalTok{(summaryTable)}
\end{Highlighting}
\end{Shaded}

\begin{verbatim}
## [1] 104   4
\end{verbatim}

\begin{Shaded}
\begin{Highlighting}[]
\CommentTok{\# Look at the top enteries}
\FunctionTok{head}\NormalTok{(summaryTable)}
\end{Highlighting}
\end{Shaded}

\begin{verbatim}
##   Age BiologicalSex Diastolic Systolic
## 1  18        Female  76.03626 119.5091
## 2  18          Male  68.47743 118.7080
## 3  19        Female  74.30795 135.8965
## 4  19          Male  71.88793 124.2671
## 5  20        Female  71.04246 123.8012
## 6  20          Male  71.27133 127.4093
\end{verbatim}

We can also use this to tabulate how many observations there are for
each category

\begin{Shaded}
\begin{Highlighting}[]
\CommentTok{\# Generate table (note here we drop subject ID from the formula)}
\NormalTok{summaryTable }\OtherTok{\textless{}{-}}\NormalTok{ reshape2}\SpecialCharTok{::}\FunctionTok{dcast}\NormalTok{(longData,}\AttributeTok{formula =}\NormalTok{ Age }\SpecialCharTok{+}\NormalTok{ BiologicalSex }\SpecialCharTok{\textasciitilde{}}\NormalTok{ BP.Type,}
                                \AttributeTok{fun.aggregate =}\NormalTok{ length,}\AttributeTok{value.var =} \StringTok{\textquotesingle{}BP\textquotesingle{}}\NormalTok{)}

\CommentTok{\# Check the dimmensions }
\FunctionTok{dim}\NormalTok{(summaryTable)}
\end{Highlighting}
\end{Shaded}

\begin{verbatim}
## [1] 104   4
\end{verbatim}

\begin{Shaded}
\begin{Highlighting}[]
\CommentTok{\# Look at the top enteries}
\FunctionTok{head}\NormalTok{(summaryTable)}
\end{Highlighting}
\end{Shaded}

\begin{verbatim}
##   Age BiologicalSex Diastolic Systolic
## 1  18        Female        10       10
## 2  18          Male         8        8
## 3  19        Female         5        5
## 4  19          Male         6        6
## 5  20        Female         5        5
## 6  20          Male         8        8
\end{verbatim}

\subsubsection{In Class Activity}\label{in-class-activity}

Working in groups, define a new variable for hypertension in our
original dataset (randomData). Here we will define hypertension as
systolic blood pressure over 130 or a diastolic blood pressure over 80.
Plot the distribution in age for individuals with and without
hypertension using boxplots.

Use the \emph{melt} function to generate boxplots of the distribution of
systolic and diastolic blood pressure in hypertensive vs.~normotensive
individuals (color should be based on hypertension status).

Use the \emph{dcast} function to generate a table summarizing the mean
age, systolic, and diastolic BP for males and females, separately, with
and without hypertension. Your table should have 4 rows. Order your
table output such that it lists values for normotensive individuals
first and hypertensive individuals second.

\end{document}
