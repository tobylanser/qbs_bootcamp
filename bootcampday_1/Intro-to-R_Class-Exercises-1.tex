% Options for packages loaded elsewhere
\PassOptionsToPackage{unicode}{hyperref}
\PassOptionsToPackage{hyphens}{url}
\documentclass[
]{article}
\usepackage{xcolor}
\usepackage[margin=1in]{geometry}
\usepackage{amsmath,amssymb}
\setcounter{secnumdepth}{-\maxdimen} % remove section numbering
\usepackage{iftex}
\ifPDFTeX
  \usepackage[T1]{fontenc}
  \usepackage[utf8]{inputenc}
  \usepackage{textcomp} % provide euro and other symbols
\else % if luatex or xetex
  \usepackage{unicode-math} % this also loads fontspec
  \defaultfontfeatures{Scale=MatchLowercase}
  \defaultfontfeatures[\rmfamily]{Ligatures=TeX,Scale=1}
\fi
\usepackage{lmodern}
\ifPDFTeX\else
  % xetex/luatex font selection
\fi
% Use upquote if available, for straight quotes in verbatim environments
\IfFileExists{upquote.sty}{\usepackage{upquote}}{}
\IfFileExists{microtype.sty}{% use microtype if available
  \usepackage[]{microtype}
  \UseMicrotypeSet[protrusion]{basicmath} % disable protrusion for tt fonts
}{}
\makeatletter
\@ifundefined{KOMAClassName}{% if non-KOMA class
  \IfFileExists{parskip.sty}{%
    \usepackage{parskip}
  }{% else
    \setlength{\parindent}{0pt}
    \setlength{\parskip}{6pt plus 2pt minus 1pt}}
}{% if KOMA class
  \KOMAoptions{parskip=half}}
\makeatother
\usepackage{color}
\usepackage{fancyvrb}
\newcommand{\VerbBar}{|}
\newcommand{\VERB}{\Verb[commandchars=\\\{\}]}
\DefineVerbatimEnvironment{Highlighting}{Verbatim}{commandchars=\\\{\}}
% Add ',fontsize=\small' for more characters per line
\usepackage{framed}
\definecolor{shadecolor}{RGB}{248,248,248}
\newenvironment{Shaded}{\begin{snugshade}}{\end{snugshade}}
\newcommand{\AlertTok}[1]{\textcolor[rgb]{0.94,0.16,0.16}{#1}}
\newcommand{\AnnotationTok}[1]{\textcolor[rgb]{0.56,0.35,0.01}{\textbf{\textit{#1}}}}
\newcommand{\AttributeTok}[1]{\textcolor[rgb]{0.13,0.29,0.53}{#1}}
\newcommand{\BaseNTok}[1]{\textcolor[rgb]{0.00,0.00,0.81}{#1}}
\newcommand{\BuiltInTok}[1]{#1}
\newcommand{\CharTok}[1]{\textcolor[rgb]{0.31,0.60,0.02}{#1}}
\newcommand{\CommentTok}[1]{\textcolor[rgb]{0.56,0.35,0.01}{\textit{#1}}}
\newcommand{\CommentVarTok}[1]{\textcolor[rgb]{0.56,0.35,0.01}{\textbf{\textit{#1}}}}
\newcommand{\ConstantTok}[1]{\textcolor[rgb]{0.56,0.35,0.01}{#1}}
\newcommand{\ControlFlowTok}[1]{\textcolor[rgb]{0.13,0.29,0.53}{\textbf{#1}}}
\newcommand{\DataTypeTok}[1]{\textcolor[rgb]{0.13,0.29,0.53}{#1}}
\newcommand{\DecValTok}[1]{\textcolor[rgb]{0.00,0.00,0.81}{#1}}
\newcommand{\DocumentationTok}[1]{\textcolor[rgb]{0.56,0.35,0.01}{\textbf{\textit{#1}}}}
\newcommand{\ErrorTok}[1]{\textcolor[rgb]{0.64,0.00,0.00}{\textbf{#1}}}
\newcommand{\ExtensionTok}[1]{#1}
\newcommand{\FloatTok}[1]{\textcolor[rgb]{0.00,0.00,0.81}{#1}}
\newcommand{\FunctionTok}[1]{\textcolor[rgb]{0.13,0.29,0.53}{\textbf{#1}}}
\newcommand{\ImportTok}[1]{#1}
\newcommand{\InformationTok}[1]{\textcolor[rgb]{0.56,0.35,0.01}{\textbf{\textit{#1}}}}
\newcommand{\KeywordTok}[1]{\textcolor[rgb]{0.13,0.29,0.53}{\textbf{#1}}}
\newcommand{\NormalTok}[1]{#1}
\newcommand{\OperatorTok}[1]{\textcolor[rgb]{0.81,0.36,0.00}{\textbf{#1}}}
\newcommand{\OtherTok}[1]{\textcolor[rgb]{0.56,0.35,0.01}{#1}}
\newcommand{\PreprocessorTok}[1]{\textcolor[rgb]{0.56,0.35,0.01}{\textit{#1}}}
\newcommand{\RegionMarkerTok}[1]{#1}
\newcommand{\SpecialCharTok}[1]{\textcolor[rgb]{0.81,0.36,0.00}{\textbf{#1}}}
\newcommand{\SpecialStringTok}[1]{\textcolor[rgb]{0.31,0.60,0.02}{#1}}
\newcommand{\StringTok}[1]{\textcolor[rgb]{0.31,0.60,0.02}{#1}}
\newcommand{\VariableTok}[1]{\textcolor[rgb]{0.00,0.00,0.00}{#1}}
\newcommand{\VerbatimStringTok}[1]{\textcolor[rgb]{0.31,0.60,0.02}{#1}}
\newcommand{\WarningTok}[1]{\textcolor[rgb]{0.56,0.35,0.01}{\textbf{\textit{#1}}}}
\usepackage{graphicx}
\makeatletter
\newsavebox\pandoc@box
\newcommand*\pandocbounded[1]{% scales image to fit in text height/width
  \sbox\pandoc@box{#1}%
  \Gscale@div\@tempa{\textheight}{\dimexpr\ht\pandoc@box+\dp\pandoc@box\relax}%
  \Gscale@div\@tempb{\linewidth}{\wd\pandoc@box}%
  \ifdim\@tempb\p@<\@tempa\p@\let\@tempa\@tempb\fi% select the smaller of both
  \ifdim\@tempa\p@<\p@\scalebox{\@tempa}{\usebox\pandoc@box}%
  \else\usebox{\pandoc@box}%
  \fi%
}
% Set default figure placement to htbp
\def\fps@figure{htbp}
\makeatother
\setlength{\emergencystretch}{3em} % prevent overfull lines
\providecommand{\tightlist}{%
  \setlength{\itemsep}{0pt}\setlength{\parskip}{0pt}}
\usepackage{bookmark}
\IfFileExists{xurl.sty}{\usepackage{xurl}}{} % add URL line breaks if available
\urlstyle{same}
\hypersetup{
  pdftitle={Intro to R},
  pdfauthor={QBS Bootcamp 2025},
  hidelinks,
  pdfcreator={LaTeX via pandoc}}

\title{Intro to R}
\author{QBS Bootcamp 2025}
\date{}

\begin{document}
\maketitle

\section{Class Exercises, 1}\label{class-exercises-1}

Please work in groups to solve the problems provided in our very first
in class assignment. If you complete this early, you are free to leave
before our class time runs out. If you are unable to complete these
exercises before the end of class, you can continue at home, though it
is not required that you do so. These will not be graded!

\begin{enumerate}
\def\labelenumi{\arabic{enumi})}
\tightlist
\item
  Below I have assigned ``things'' to variables. Using code, can you
  tell me the type of data of each ``thing'' is?
\end{enumerate}

\begin{Shaded}
\begin{Highlighting}[]
\NormalTok{a }\OtherTok{\textless{}{-}} \FloatTok{3.14}
\NormalTok{b }\OtherTok{\textless{}{-}} \StringTok{"pie"}
\NormalTok{c }\OtherTok{\textless{}{-}} \DecValTok{3}

\CommentTok{\# Check the data types}
\FunctionTok{class}\NormalTok{(a)}
\end{Highlighting}
\end{Shaded}

\begin{verbatim}
## [1] "numeric"
\end{verbatim}

\begin{Shaded}
\begin{Highlighting}[]
\FunctionTok{class}\NormalTok{(b)}
\end{Highlighting}
\end{Shaded}

\begin{verbatim}
## [1] "character"
\end{verbatim}

\begin{Shaded}
\begin{Highlighting}[]
\FunctionTok{class}\NormalTok{(c)}
\end{Highlighting}
\end{Shaded}

\begin{verbatim}
## [1] "numeric"
\end{verbatim}

\begin{enumerate}
\def\labelenumi{\arabic{enumi})}
\setcounter{enumi}{1}
\tightlist
\item
  Create a nested list in the code chunk below. This list should contain
  one vector of integers, one vector of float numbers, and one vector of
  strings or characters.
\end{enumerate}

\begin{Shaded}
\begin{Highlighting}[]
\CommentTok{\# Create a nested list with different data types}
\NormalTok{nested\_list }\OtherTok{\textless{}{-}} \FunctionTok{list}\NormalTok{(}
    \AttributeTok{integers =} \FunctionTok{c}\NormalTok{(}\DecValTok{1}\NormalTok{, }\DecValTok{2}\NormalTok{, }\DecValTok{3}\NormalTok{, }\DecValTok{4}\NormalTok{, }\DecValTok{5}\NormalTok{),}
    \AttributeTok{floats =} \FunctionTok{c}\NormalTok{(}\FloatTok{1.5}\NormalTok{, }\FloatTok{2.7}\NormalTok{, }\FloatTok{3.14}\NormalTok{, }\FloatTok{4.2}\NormalTok{, }\FloatTok{5.9}\NormalTok{),}
    \AttributeTok{strings =} \FunctionTok{c}\NormalTok{(}\StringTok{"apple"}\NormalTok{, }\StringTok{"banana"}\NormalTok{, }\StringTok{"cherry"}\NormalTok{, }\StringTok{"date"}\NormalTok{, }\StringTok{"elderberry"}\NormalTok{)}
\NormalTok{)}

\CommentTok{\# Display the nested list}
\NormalTok{nested\_list}
\end{Highlighting}
\end{Shaded}

\begin{verbatim}
## $integers
## [1] 1 2 3 4 5
## 
## $floats
## [1] 1.50 2.70 3.14 4.20 5.90
## 
## $strings
## [1] "apple"      "banana"     "cherry"     "date"       "elderberry"
\end{verbatim}

\begin{enumerate}
\def\labelenumi{\arabic{enumi})}
\setcounter{enumi}{2}
\tightlist
\item
  I have provided you with a vector of float numbers. Please convert
  these float numbers to integers and save the output vector to a vector
  called ``my\_integers''.
\end{enumerate}

\begin{Shaded}
\begin{Highlighting}[]
\NormalTok{my\_floats }\OtherTok{\textless{}{-}} \FunctionTok{c}\NormalTok{(}\FloatTok{1.67}\NormalTok{, }\FloatTok{1.11}\NormalTok{, }\FloatTok{2.25}\NormalTok{, }\FloatTok{8.88}\NormalTok{, }\FloatTok{6.67}\NormalTok{, }\FloatTok{1048.2}\NormalTok{)}

\CommentTok{\# Convert floats to integers}
\NormalTok{my\_integers }\OtherTok{\textless{}{-}} \FunctionTok{as.integer}\NormalTok{(my\_floats)}

\CommentTok{\# Display the result}
\NormalTok{my\_integers}
\end{Highlighting}
\end{Shaded}

\begin{verbatim}
## [1]    1    1    2    8    6 1048
\end{verbatim}

\begin{enumerate}
\def\labelenumi{\arabic{enumi})}
\setcounter{enumi}{3}
\tightlist
\item
  The vector of words was taken from a 2020 Science article about
  rheumatoid arthritis (RA) associated joint damage. If you are
  interested in accessing the whole article it can be found
  \href{https://www.science.org/doi/10.1126/sciadv.abd2688}{here}. Using
  this vector of words and your recently acquired programming skills,
  select the word ``rheumatoid'' and print it out.
\end{enumerate}

\begin{Shaded}
\begin{Highlighting}[]
\NormalTok{arthritis }\OtherTok{\textless{}{-}} \FunctionTok{c}\NormalTok{(}\StringTok{"Formation"}\NormalTok{, }\StringTok{"of"}\NormalTok{, }\StringTok{"autoantibodies"}\NormalTok{, }\StringTok{"to"}\NormalTok{, }\StringTok{"carbamylated"}\NormalTok{, }\StringTok{"proteins"}\NormalTok{, }\StringTok{"(anti{-}CarP)"}\NormalTok{, }\StringTok{"is"}\NormalTok{, }\StringTok{"considered"}\NormalTok{, }\StringTok{"detrimental"}\NormalTok{, }\StringTok{"in"}\NormalTok{, }\StringTok{"the"}\NormalTok{, }\StringTok{"prognosis"}\NormalTok{, }\StringTok{"of"}\NormalTok{, }\StringTok{"erosive"}\NormalTok{, }\StringTok{"rheumatoid"}\NormalTok{, }\StringTok{"arthritis"}\NormalTok{, }\StringTok{"(RA)."}\NormalTok{, }\StringTok{"The"}\NormalTok{, }\StringTok{"source"}\NormalTok{, }\StringTok{"of"}\NormalTok{, }\StringTok{"carbamylated"}\NormalTok{, }\StringTok{"antigens"}\NormalTok{, }\StringTok{"and"}\NormalTok{, }\StringTok{"the"}\NormalTok{, }\StringTok{"mechanisms"}\NormalTok{, }\StringTok{"by"}\NormalTok{, }\StringTok{"which"}\NormalTok{, }\StringTok{"anti{-}CarP"}\NormalTok{, }\StringTok{"antibodies"}\NormalTok{, }\StringTok{"promote"}\NormalTok{, }\StringTok{"bone"}\NormalTok{, }\StringTok{"erosion"}\NormalTok{, }\StringTok{"in"}\NormalTok{, }\StringTok{"RA"}\NormalTok{, }\StringTok{"remain"}\NormalTok{, }\StringTok{"unknown."}\NormalTok{, }\ConstantTok{NA}\NormalTok{, }\StringTok{"Here,"}\NormalTok{, }\StringTok{"we"}\NormalTok{, }\StringTok{"find"}\NormalTok{, }\StringTok{"that"}\NormalTok{, }\StringTok{"neutrophil"}\NormalTok{, }\StringTok{"extracellular"}\NormalTok{, }\StringTok{"traps"}\NormalTok{, }\StringTok{"(NETs)"}\NormalTok{, }\StringTok{"externalize"}\NormalTok{, }\StringTok{"carbamylated"}\NormalTok{, }\StringTok{"proteins"}\NormalTok{, }\StringTok{"and"}\NormalTok{, }\StringTok{"that"}\NormalTok{, }\StringTok{"RA"}\NormalTok{, }\StringTok{"subjects"}\NormalTok{, }\StringTok{"develop"}\NormalTok{, }\StringTok{"autoantibodies"}\NormalTok{, }\StringTok{"against"}\NormalTok{, }\StringTok{"carbamylated"}\NormalTok{, }\StringTok{"NET"}\NormalTok{, }\StringTok{"(cNET)"}\NormalTok{, }\StringTok{"antigens"}\NormalTok{, }\StringTok{"that,"}\NormalTok{, }\StringTok{"in"}\NormalTok{, }\StringTok{"turn,"}\NormalTok{, }\StringTok{"correlate"}\NormalTok{, }\StringTok{"with"}\NormalTok{, }\StringTok{"levels"}\NormalTok{, }\StringTok{"of"}\NormalTok{, }\StringTok{"anti{-}CarP."}\NormalTok{, }\StringTok{"Transgenic"}\NormalTok{, }\StringTok{"mice"}\NormalTok{, }\StringTok{"expressing"}\NormalTok{, }\StringTok{"the"}\NormalTok{, }\StringTok{"human"}\NormalTok{, }\StringTok{"RA"}\NormalTok{, }\StringTok{"shared"}\NormalTok{, }\StringTok{"epitope"}\NormalTok{, }\StringTok{"(HLADRB1*"}\NormalTok{, }\StringTok{"04:01)"}\NormalTok{, }\StringTok{"immunized"}\NormalTok{, }\StringTok{"with"}\NormalTok{, }\StringTok{"cNETs"}\NormalTok{, }\StringTok{"develop"}\NormalTok{, }\StringTok{"antibodies"}\NormalTok{, }\StringTok{"to"}\NormalTok{, }\StringTok{"citrullinated"}\NormalTok{, }\StringTok{"and"}\NormalTok{, }\ConstantTok{NA}\NormalTok{, }\StringTok{"carbamylated"}\NormalTok{, }\StringTok{"proteins."}\NormalTok{, }\StringTok{"Furthermore,"}\NormalTok{, }\StringTok{"anti–carbamylated"}\NormalTok{, }\StringTok{"histone"}\NormalTok{, }\StringTok{"antibodies"}\NormalTok{, }\StringTok{"correlate"}\NormalTok{, }\StringTok{"with"}\NormalTok{, }\StringTok{"radiographic"}\NormalTok{, }\StringTok{"bone"}\NormalTok{, }\StringTok{"erosion"}\NormalTok{, }\StringTok{"in"}\NormalTok{, }\StringTok{"RA"}\NormalTok{, }\StringTok{"subjects."}\NormalTok{, }\StringTok{"Moreover,"}\NormalTok{, }\StringTok{"anti–carbamylated"}\NormalTok{, }\StringTok{"histone–immunoglobulin"}\NormalTok{, }\StringTok{"G"}\NormalTok{, }\StringTok{"immune"}\NormalTok{, }\StringTok{"complexes"}\NormalTok{, }\StringTok{"promote"}\NormalTok{, }\StringTok{"osteoclast"}\NormalTok{, }\StringTok{"differentiation"}\NormalTok{, }\StringTok{"and"}\NormalTok{, }\StringTok{"potentiate"}\NormalTok{, }\StringTok{"osteoclast{-}mediated"}\NormalTok{, }\ConstantTok{NA}\NormalTok{, }\StringTok{"matrix"}\NormalTok{, }\StringTok{"resorption."}\NormalTok{, }\StringTok{"These"}\NormalTok{, }\StringTok{"results"}\NormalTok{, }\StringTok{"demonstrate"}\NormalTok{, }\StringTok{"that"}\NormalTok{, }\StringTok{"carbamylated"}\NormalTok{, }\StringTok{"proteins"}\NormalTok{, }\StringTok{"present"}\NormalTok{, }\StringTok{"in"}\NormalTok{, }\StringTok{"NETs"}\NormalTok{, }\StringTok{"enhance"}\NormalTok{, }\StringTok{"pathogenic"}\NormalTok{, }\StringTok{"immune"}\NormalTok{, }\StringTok{"responses"}\NormalTok{, }\StringTok{"and"}\NormalTok{, }\StringTok{"bone"}\NormalTok{, }\StringTok{"destruction,"}\NormalTok{, }\StringTok{"which"}\NormalTok{, }\StringTok{"may"}\NormalTok{, }\StringTok{"explain"}\NormalTok{, }\StringTok{"the"}\NormalTok{, }\StringTok{"association"}\NormalTok{, }\StringTok{"between"}\NormalTok{, }\StringTok{"anti{-}CarP"}\NormalTok{, }\StringTok{"and"}\NormalTok{, }\StringTok{"erosive"}\NormalTok{, }\ConstantTok{NA}\NormalTok{, }\StringTok{"arthritis"}\NormalTok{, }\StringTok{"in"}\NormalTok{, }\StringTok{"RA."}\NormalTok{)}

\CommentTok{\# Find and print the word "rheumatoid"}
\NormalTok{rheumatoid\_index }\OtherTok{\textless{}{-}} \FunctionTok{which}\NormalTok{(arthritis }\SpecialCharTok{==} \StringTok{"rheumatoid"}\NormalTok{)}
\NormalTok{arthritis[rheumatoid\_index]}
\end{Highlighting}
\end{Shaded}

\begin{verbatim}
## [1] "rheumatoid"
\end{verbatim}

\begin{enumerate}
\def\labelenumi{\arabic{enumi})}
\setcounter{enumi}{4}
\tightlist
\item
  From the vector in the last question, count how many NA's are present.
  Use coding!
\end{enumerate}

\begin{Shaded}
\begin{Highlighting}[]
\CommentTok{\# Count NA values in the arthritis vector}
\NormalTok{na\_count }\OtherTok{\textless{}{-}} \FunctionTok{sum}\NormalTok{(}\FunctionTok{is.na}\NormalTok{(arthritis))}
\NormalTok{na\_count}
\end{Highlighting}
\end{Shaded}

\begin{verbatim}
## [1] 4
\end{verbatim}

\begin{enumerate}
\def\labelenumi{\arabic{enumi})}
\setcounter{enumi}{5}
\tightlist
\item
  Count how many numbers in the below vector are less than 10. Use
  coding!
\end{enumerate}

\begin{Shaded}
\begin{Highlighting}[]
\NormalTok{number\_vector }\OtherTok{\textless{}{-}} \FunctionTok{c}\NormalTok{(}\DecValTok{30}\NormalTok{, }\DecValTok{29}\NormalTok{, }\DecValTok{48}\NormalTok{, }\DecValTok{10}\NormalTok{, }\DecValTok{0}\NormalTok{, }\DecValTok{8}\NormalTok{, }\DecValTok{56}\NormalTok{, }\DecValTok{77}\NormalTok{, }\DecValTok{211}\NormalTok{, }\DecValTok{674}\NormalTok{, }\DecValTok{1}\NormalTok{)}

\CommentTok{\# Count numbers less than 10}
\NormalTok{less\_than\_10 }\OtherTok{\textless{}{-}} \FunctionTok{sum}\NormalTok{(number\_vector }\SpecialCharTok{\textless{}} \DecValTok{10}\NormalTok{)}
\NormalTok{less\_than\_10}
\end{Highlighting}
\end{Shaded}

\begin{verbatim}
## [1] 3
\end{verbatim}

\begin{enumerate}
\def\labelenumi{\arabic{enumi})}
\setcounter{enumi}{6}
\tightlist
\item
  Count how many times ``a'' is present in the below vector. Use coding!
\end{enumerate}

\begin{Shaded}
\begin{Highlighting}[]
\NormalTok{letter\_vector }\OtherTok{\textless{}{-}} \FunctionTok{c}\NormalTok{(}\StringTok{"v"}\NormalTok{, }\StringTok{"a"}\NormalTok{, }\StringTok{"b"}\NormalTok{, }\StringTok{"g"}\NormalTok{, }\StringTok{"f"}\NormalTok{, }\StringTok{"a"}\NormalTok{, }\StringTok{"n"}\NormalTok{, }\StringTok{"m"}\NormalTok{, }\StringTok{"q"}\NormalTok{, }\StringTok{"a"}\NormalTok{, }\StringTok{"c"}\NormalTok{, }\StringTok{"w"}\NormalTok{, }\StringTok{"w"}\NormalTok{, }\StringTok{"i"}\NormalTok{, }\StringTok{"e"}\NormalTok{, }\StringTok{"y"}\NormalTok{)}

\CommentTok{\# Count occurrences of "a"}
\NormalTok{a\_count }\OtherTok{\textless{}{-}} \FunctionTok{sum}\NormalTok{(letter\_vector }\SpecialCharTok{==} \StringTok{"a"}\NormalTok{)}
\NormalTok{a\_count}
\end{Highlighting}
\end{Shaded}

\begin{verbatim}
## [1] 3
\end{verbatim}

\begin{enumerate}
\def\labelenumi{\arabic{enumi})}
\setcounter{enumi}{7}
\tightlist
\item
  Count how many number in the below vector are equal to 5 or greater
  than 5555.
\end{enumerate}

\begin{Shaded}
\begin{Highlighting}[]
\NormalTok{number\_vector }\OtherTok{\textless{}{-}} \FunctionTok{c}\NormalTok{(}\DecValTok{5}\NormalTok{, }\DecValTok{5}\NormalTok{, }\DecValTok{55}\NormalTok{, }\DecValTok{555555}\NormalTok{, }\DecValTok{555}\NormalTok{, }\DecValTok{55555}\NormalTok{, }\DecValTok{5}\NormalTok{, }\DecValTok{5}\NormalTok{, }\DecValTok{555555555}\NormalTok{, }\DecValTok{55}\NormalTok{, }\DecValTok{55}\NormalTok{, }\DecValTok{555}\NormalTok{, }\DecValTok{5}\NormalTok{, }\DecValTok{5}\NormalTok{, }\DecValTok{55555555}\NormalTok{, }\DecValTok{5}\NormalTok{, }\DecValTok{55}\NormalTok{, }\DecValTok{5}\NormalTok{, }\DecValTok{5555}\NormalTok{, }\DecValTok{5}\NormalTok{, }\DecValTok{555}\NormalTok{, }\DecValTok{5555}\NormalTok{)}

\CommentTok{\# Count numbers equal to 5 or greater than 5555}
\NormalTok{count\_condition }\OtherTok{\textless{}{-}} \FunctionTok{sum}\NormalTok{(number\_vector }\SpecialCharTok{==} \DecValTok{5} \SpecialCharTok{|}\NormalTok{ number\_vector }\SpecialCharTok{\textgreater{}} \DecValTok{5555}\NormalTok{)}
\NormalTok{count\_condition}
\end{Highlighting}
\end{Shaded}

\begin{verbatim}
## [1] 13
\end{verbatim}

\begin{enumerate}
\def\labelenumi{\arabic{enumi})}
\setcounter{enumi}{8}
\tightlist
\item
  Make a vector containing any 5 items that you want. These can be of
  type integer, numeric, or character. Subset your vector to contain
  only 3 items.
\end{enumerate}

\begin{Shaded}
\begin{Highlighting}[]
\CommentTok{\# Create a vector with 5 items}
\NormalTok{my\_vector }\OtherTok{\textless{}{-}} \FunctionTok{c}\NormalTok{(}\StringTok{"apple"}\NormalTok{, }\DecValTok{42}\NormalTok{, }\FloatTok{3.14}\NormalTok{, }\StringTok{"banana"}\NormalTok{, }\DecValTok{100}\NormalTok{)}

\CommentTok{\# Subset to contain only 3 items (first 3)}
\NormalTok{subset\_vector }\OtherTok{\textless{}{-}}\NormalTok{ my\_vector[}\DecValTok{1}\SpecialCharTok{:}\DecValTok{3}\NormalTok{]}
\NormalTok{subset\_vector}
\end{Highlighting}
\end{Shaded}

\begin{verbatim}
## [1] "apple" "42"    "3.14"
\end{verbatim}

\begin{enumerate}
\def\labelenumi{\arabic{enumi})}
\setcounter{enumi}{9}
\tightlist
\item
  Report the factors of the levels in the below vector.
\end{enumerate}

\begin{Shaded}
\begin{Highlighting}[]
\NormalTok{smoking\_status }\OtherTok{\textless{}{-}} \FunctionTok{c}\NormalTok{(}\StringTok{"Never Smoker"}\NormalTok{, }\StringTok{"Current Smoker"}\NormalTok{, }\StringTok{"Former Smoker"}\NormalTok{, }\StringTok{"Former Smoker"}\NormalTok{, }\StringTok{"Never Smoker"}\NormalTok{, }\StringTok{"Never Smoker"}\NormalTok{, }\StringTok{"Never Smoker"}\NormalTok{, }\StringTok{"Current Smoker"}\NormalTok{, }\StringTok{"Former Smoker"}\NormalTok{, }\StringTok{"Former Smoker"}\NormalTok{)}

\CommentTok{\# Convert to factor and get levels}
\NormalTok{smoking\_factor }\OtherTok{\textless{}{-}} \FunctionTok{factor}\NormalTok{(smoking\_status)}
\FunctionTok{levels}\NormalTok{(smoking\_factor)}
\end{Highlighting}
\end{Shaded}

\begin{verbatim}
## [1] "Current Smoker" "Former Smoker"  "Never Smoker"
\end{verbatim}

\begin{enumerate}
\def\labelenumi{\arabic{enumi})}
\setcounter{enumi}{10}
\tightlist
\item
  Below I have supplied you with a list of vectors containing the names
  of foods. Add logical names to each of the vectors in the list (eg.
  ``Meats'',``Veggies'', ``Grains'',``Fruit''). Access the second vector
  in the lists. Return the type of items in this vector.
\end{enumerate}

\begin{Shaded}
\begin{Highlighting}[]
\NormalTok{grocery\_list }\OtherTok{\textless{}{-}} \FunctionTok{list}\NormalTok{(}
    \FunctionTok{c}\NormalTok{(}\StringTok{"Peaches"}\NormalTok{, }\StringTok{"Bananas"}\NormalTok{, }\StringTok{"Strawberries"}\NormalTok{, }\StringTok{"Melon"}\NormalTok{),}
    \FunctionTok{c}\NormalTok{(}\StringTok{"Spinach"}\NormalTok{, }\StringTok{"Lettuce"}\NormalTok{, }\StringTok{"Carrot"}\NormalTok{, }\StringTok{"Kale"}\NormalTok{),}
    \FunctionTok{c}\NormalTok{(}\StringTok{"Oreos"}\NormalTok{, }\StringTok{"Chocolate Cake"}\NormalTok{, }\StringTok{"Gummy Bears"}\NormalTok{, }\StringTok{"Marshmellows"}\NormalTok{)}
\NormalTok{)}

\CommentTok{\# Add names to the list}
\FunctionTok{names}\NormalTok{(grocery\_list) }\OtherTok{\textless{}{-}} \FunctionTok{c}\NormalTok{(}\StringTok{"Fruits"}\NormalTok{, }\StringTok{"Vegetables"}\NormalTok{, }\StringTok{"Sweets"}\NormalTok{)}

\CommentTok{\# Access the second vector (Vegetables)}
\NormalTok{second\_vector }\OtherTok{\textless{}{-}}\NormalTok{ grocery\_list[[}\DecValTok{2}\NormalTok{]]}
\NormalTok{second\_vector}
\end{Highlighting}
\end{Shaded}

\begin{verbatim}
## [1] "Spinach" "Lettuce" "Carrot"  "Kale"
\end{verbatim}

\begin{Shaded}
\begin{Highlighting}[]
\CommentTok{\# Check the type of items in the second vector}
\FunctionTok{class}\NormalTok{(second\_vector)}
\end{Highlighting}
\end{Shaded}

\begin{verbatim}
## [1] "character"
\end{verbatim}

\end{document}
